\documentclass[serif, aspectratio=169]{beamer}
\usepackage[T1]{fontenc} 
\usepackage{fourier}
\usepackage{hyperref}
\usepackage{latexsym,amsmath,xcolor,multicol,booktabs,calligra}
\usepackage{graphicx,pstricks,listings,stackengine}
\usepackage{listings}

\author{Dr.Hajialiasgari}
\title{Introduction To Python}
\institute{
    Tehran University \\
    Of\\
    Medical Science
}
\date{\small \today}
\usepackage{UoWstyle}

% Define custom colors and styles for listings
\definecolor{deepblue}{rgb}{0,0,0.5}
\definecolor{deepred}{RGB}{153,0,0}
\definecolor{deepgreen}{rgb}{0,0.5,0}
\definecolor{halfgray}{gray}{0.55}

\lstset{
    basicstyle=\ttfamily\small,
    keywordstyle=\bfseries\color{deepblue},
    emphstyle=\ttfamily\color{deepred},
    stringstyle=\color{deepgreen},
    numbers=left,
    numberstyle=\small\color{halfgray},
    rulesepcolor=\color{red!20!green!20!blue!20},
    frame=shadowbox,
}

\begin{document}

\begin{frame}
    \titlepage
    \vspace*{-0.6cm}
    \begin{figure}[htpb]
        \begin{center}
            \includegraphics[keepaspectratio, scale=0.05]{Tumsl-logo.png}
        \end{center}
    \end{figure}
\end{frame}

\begin{frame}    
\tableofcontents[sectionstyle=show, subsectionstyle=show/shaded/hide, subsubsectionstyle=show/shaded/hide]
\end{frame}

\section{Strings}

\begin{frame}{String Data Type}
    \begin{itemize}
        \item A string is a sequence of characters.
        \item A string literal uses quotes: \texttt{'Hello'} or \texttt{"Hello"}.
        \item For strings, \texttt{+} means “concatenate.”
        \item When a string contains numbers, it is still a string.
        \item We can convert numbers in a string into a number using \texttt{int()}.
    \end{itemize}
\end{frame}

\begin{frame}[fragile]{String Example Code}
    \begin{lstlisting}
str1 = "Hello"
str2 = 'World!'
str3 = str1 + str2
print(str3)  # Output: HelloWorld!

str4 = '123'
str4 = str4 + 1  # TypeError: cannot concatenate 'str' and 'int' objects

x = int(str4) + 1
print(x)  # Output: 124
    \end{lstlisting}
\end{frame}

\begin{frame}{Looking Inside Strings}
    \begin{itemize}
        \item Access any single character in a string using an index specified in square brackets.
        \item Index value must be an integer and starts at zero.
        \item Index value can be an expression that is computed.
    \end{itemize}
\end{frame}

\begin{frame}[fragile]{Indexing Example}
    \begin{lstlisting}
fruit = 'banana'
letter = fruit[1]
print(letter)  # Output: a

x = 3
w = fruit[x - 1]
print(w)  # Output: n
    \end{lstlisting}
\end{frame}

\begin{frame}{A Character Too Far}
    \begin{itemize}
        \item Accessing beyond the end of a string results in an error.
        \item Be careful when constructing index values and slices.
    \end{itemize}
\end{frame}

\begin{frame}[fragile]{IndexError Example}
    \begin{lstlisting}
zot = 'abc'
print(zot[5])  # IndexError: string index out of range
    \end{lstlisting}
\end{frame}

\begin{frame}{Strings Have Length}
    \begin{itemize}
        \item The built-in function \texttt{len} gives the length of a string.
    \end{itemize}
\end{frame}

\begin{frame}{Indexing Example}
    \begin{table}[]
        \centering
        \begin{tabular}{c|c|c|c|c|c|c}
            \textbf{Index} & 0 & 1 & 2 & 3 & 4 & 5 \\
            \hline
            \textbf{Character} & b & a & n & a & n & a \\
        \end{tabular}
        \caption{Indexing of Each Character in "banana"}
    \end{table}
    \vspace{0.5cm}
    \begin{itemize}
        \item Each character in the word "banana" has a unique index starting from 0.
        \item Accessing \texttt{fruit[1]} will return "a".
    \end{itemize}
\end{frame}

\begin{frame}[fragile]{Length Example}
    \begin{lstlisting}
fruit = 'banana'
print(len(fruit))  # Output: 6
    \end{lstlisting}
\end{frame}

\begin{frame}{Looping Through Strings}
    \begin{itemize}
        \item Using a \texttt{while} statement, iteration variable, and \texttt{len} function to construct a loop.
    \end{itemize}
\end{frame}

\begin{frame}[fragile]{Looping Example}
    \begin{lstlisting}
fruit = 'banana'
for letter in fruit: 
    print(letter)
    
# Output
# 0 b
# 1 a
# 2 n
# 3 a
# 4 n
# 5 a
    \end{lstlisting}
\end{frame}

\begin{frame}{Looping and Counting}
    \begin{itemize}
        \item Count the number of times a character appears in a string.
    \end{itemize}
\end{frame}

\begin{frame}[fragile]{Counting Example}
    \begin{lstlisting}
word = 'banana'
count = 0
for letter in word:
    if letter == 'a':
        count += 1
print(count)  # Output: 3
    \end{lstlisting}
\end{frame}

\begin{frame}{Slicing Strings}
    \begin{itemize}
        \item Use a colon operator to access a continuous section of a string.
        \item The second number is “up to but not including.”
    \end{itemize}
\end{frame}

\begin{frame}[fragile]{Slicing Example}
    \begin{lstlisting}
s = 'Monty Python'
print(s[0:4])  # Output: Mont
print(s[6:7])  # Output: P
print(s[6:20]) # Output: Python
    \end{lstlisting}
\end{frame}

\begin{frame}{String Concatenation}
    \begin{itemize}
        \item The \texttt{+} operator is used for concatenation.
    \end{itemize}
\end{frame}

\begin{frame}[fragile]{Concatenation Example}
    \begin{lstlisting}
a = 'Hello'
b = a + 'There'
print(b)  # Output: HelloThere

c = a + ' ' + 'There'
print(c)  # Output: Hello There
    \end{lstlisting}
\end{frame}

\begin{frame}{Using \texttt{in} as a Logical Operator}
    \begin{itemize}
        \item The \texttt{\color{red}in} keyword checks if one string is in another.
        \item Returns \texttt{True} or \texttt{False}.
    \end{itemize}
\end{frame}

\begin{frame}[fragile]{\texttt{in} Example}
    \begin{lstlisting}
fruit = 'banana'
print('n' in fruit)  # Output: True
print('m' in fruit)  # Output: False
print('nan' in fruit) # Output: True

if 'a' in fruit:
    print('Found it!')
    \end{lstlisting}
\end{frame}

\begin{frame}{String Library}
    \begin{itemize}
        \item Python has many built-in string functions.
        \item Functions are invoked by appending them to the string variable.
    \end{itemize}
\end{frame}

\begin{frame}[fragile]{String Functions Example}
    \begin{lstlisting}
greet = 'Hello Bob'
zap = greet.lower()
print(zap)        # Output: hello bob
print(greet)      # Output: Hello Bob
print('Hi There'.lower())  # Output: hi there
    \end{lstlisting}
\end{frame}

\begin{frame}{Searching a String}
    \begin{itemize}
        \item Use \texttt{find()} to search for a substring.
        \item Returns the index of the first occurrence, or -1 if not found.
    \end{itemize}
\end{frame}

\begin{frame}[fragile]{Search Example}
    \begin{lstlisting}
fruit = 'banana'
pos = fruit.find('na')
print(pos)  # Output: 2

aa = fruit.find('z')
print(aa)  # Output: -1
    \end{lstlisting}
\end{frame}

\begin{frame}{Search and Replace}
    \begin{itemize}
        \item The \texttt{replace()} function replaces occurrences of a substring.
    \end{itemize}
\end{frame}

\begin{frame}[fragile]{Replace Example}
    \begin{lstlisting}
greet = 'Hello Bob'
nstr = greet.replace('Bob', 'Jane')
print(nstr)  # Output: Hello Jane

nstr = greet.replace('o', 'X')
print(nstr)  # Output: HellX BXb
    \end{lstlisting}
\end{frame}

\begin{frame}{Stripping Whitespace}
    \begin{itemize}
        \item \texttt{lstrip()}, \texttt{rstrip()}, and \texttt{strip()} remove whitespace from strings.
    \end{itemize}
\end{frame}

\begin{frame}[fragile]{Stripping Example}
    \begin{lstlisting}
greet = '   Hello Bob  '
print(greet.lstrip())  # Output: 'Hello Bob  '
print(greet.rstrip())  # Output: '   Hello Bob'
print(greet.strip())   # Output: 'Hello Bob'
    \end{lstlisting}
\end{frame}

\begin{frame}{Prefixes}
    \begin{itemize}
        \item Use \texttt{startswith()} to check if a string starts with a specific prefix.
    \end{itemize}
\end{frame}

\begin{frame}[fragile]{Prefix Example}
    \begin{lstlisting}
line = 'Please have a nice day'
print(line.startswith('Please'))  # Output: True
print(line.startswith('p'))       # Output: False
    \end{lstlisting}
\end{frame}

\begin{frame}
    \begin{center}
        {\Huge\ End of Strings}
    \end{center}
\end{frame}

\end{document}
