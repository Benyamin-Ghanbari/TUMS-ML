\documentclass[serif, aspectratio=169]{beamer}
\usepackage[T1]{fontenc} 
\usepackage{fourier}
\usepackage{hyperref}
\usepackage{latexsym,amsmath,xcolor,multicol,booktabs,calligra}
\usepackage{graphicx,pstricks,listings,stackengine}
\usepackage{listings}

\author{Dr.Hajialiasgari}
\title{Introduction To Python}
\institute{
    Tehran University \\
    Of\\
    Medical Science
}
\date{\small \today}
\usepackage{UoWstyle}

% Define custom colors and styles for listings
\definecolor{deepblue}{rgb}{0,0,0.5}
\definecolor{deepred}{RGB}{153,0,0}
\definecolor{deepgreen}{rgb}{0,0.5,0}
\definecolor{halfgray}{gray}{0.55}

\lstset{
    basicstyle=\ttfamily\small,
    keywordstyle=\bfseries\color{deepblue},
    emphstyle=\ttfamily\color{deepred},
    stringstyle=\color{deepgreen},
    numbers=left,
    numberstyle=\small\color{halfgray},
    rulesepcolor=\color{red!20!green!20!blue!20},
    frame=shadowbox,
}

\begin{document}

\begin{frame}
    \titlepage
    \vspace*{-0.6cm}
    \begin{figure}[htpb]
        \begin{center}
            \includegraphics[keepaspectratio, scale=0.05]{Tumsl-logo.png}
        \end{center}
    \end{figure}
\end{frame}

\begin{frame}    
\tableofcontents[sectionstyle=show, subsectionstyle=show/shaded/hide, subsubsectionstyle=show/shaded/hide]
\end{frame}

\section{Python Lists}

\begin{frame}{What is Not a “Collection”?}
    \begin{itemize}
        \item Most of our variables have one value in them .
        \item When we put a new value in the variable, the old value is overwritten .
    \end{itemize}
\end{frame}

\begin{frame}[fragile]{Code Example}
    \begin{lstlisting}
$ python
>>> x = 2
>>> x = 4
>>> print(x)
4
    \end{lstlisting}
\end{frame}

\begin{frame}{A List is a Kind of Collection}
    \begin{itemize}
        \item A collection allows us to put many values in a single “variable”.
        \item A collection is nice because we can carry all many values around in one convenient package.
    \end{itemize}
\end{frame}

\begin{frame}[fragile]{List Example}
    \begin{lstlisting}
friends = [ 'Joseph', 'Glenn', 'Sally' ]
carryon = [ 'socks', 'shirt', 'perfume' ]

    \end{lstlisting}
\end{frame}

\begin{frame}{A Character Too Far}
    \begin{itemize}
        \item Accessing beyond the end of a string results in an error.
        \item Be careful when constructing index values and slices.
    \end{itemize}
\end{frame}

\begin{frame}{List Constants}
    \begin{itemize}
        \item List constants are surrounded by square brackets and the elements in the list are separated by commas
        \item A list element can be any Python object - even another list
        \item A list can be empty
    \end{itemize}
\end{frame}

\begin{frame}[fragile]{Code Example}
    \begin{lstlisting}
>>> print([1, 24, 76])
[1, 24, 76]
>>> print(['red', 'yellow', 'blue'])
['red', 'yellow', 'blue']
>>> print(['red', 24, 98.6])
['red', 24, 98.6]
>>> print([ 1, [5, 6], 7])
[1, [5, 6], 7]
>>> print([])
[]
    \end{lstlisting}
\end{frame}

\begin{frame}[fragile]{Lists and Definite Loop }
    \begin{lstlisting}
friends = ['Joseph', 'Glenn', 'Sally']
for friend in friends :
    print('Happy New Year:',  friend)
print('Done!')

#output 

Happy New Year: Joseph
Happy New Year: Glenn
Happy New Year: Sally
Done!
    \end{lstlisting}
\end{frame}

\begin{frame}{Looking Inside Lists}
    \begin{itemize}
        \item Just like strings, we can get at any single element in a list using an index specified in square brackets
    \end{itemize}
\end{frame}

\begin{frame}[fragile]{Code Example}
    \begin{lstlisting}

>>> friends = [ 'Joseph', 'Glenn', 'Sally' ]
>>> print(friends[1])
Glenn
>>> 
    \end{lstlisting}
\end{frame}

\begin{frame}{Lists are Mutable}
    \begin{itemize}
        \item Strings are “immutable” - we cannot change the contents of a string - we must make a new string to make any change
        \item Lists are “mutable” - we can change an element of a list using the index operator


    \end{itemize}
\end{frame}

\begin{frame}[fragile]{Code Example}
    \begin{lstlisting}
>>> fruit = 'Banana'
>>> fruit[0] = 'b'
Traceback 
TypeError: 'str' object does not 
support item assignment
>>> x = fruit.lower()
>>> print(x)
banana
>>> lotto = [2, 14, 26, 41, 63]
>>> print(lotto)
[2, 14, 26, 41, 63]
>>> lotto[2] = 28
>>> print(lotto)
[2, 14, 28, 41, 63]

    \end{lstlisting}
\end{frame}

\begin{frame}{How Long is a List?}
    \begin{itemize}
        \item The \texttt{\color{red}len()}function takes a list as a parameter and returns the number of elements in the list
        \item Actually \texttt{\color{red}len()}tells us the number of elements of any set or sequence (such as a string...)
    \end{itemize}
\end{frame}

\begin{frame}[fragile]{Len Example}
    \begin{lstlisting}
>>> greet = 'Hello Bob'
>>> print(len(greet))
9
>>> x = [ 1, 2, 'joe', 99]
>>> print(len(x))
4
>>> 
    \end{lstlisting}
\end{frame}

\begin{frame}{Using the \texttt{range} Function}
    \begin{itemize}
        \item The \texttt{\color{red}range} function returns a list of numbers that range from zero to one less than the parameter
        \item We can construct an index loop using for and an integer iterator
    \end{itemize}
\end{frame}

\begin{frame}[fragile]{\texttt{range} Example}
    \begin{lstlisting}
>>> print(range(4))
[0, 1, 2, 3]
>>> friends = ['Joseph', 'Glenn', 'Sally']
>>> print(len(friends))
3
>>> print(list(range(len(friends))))
[0, 1, 2]
>>> 
    \end{lstlisting}
\end{frame}

\begin{frame}[fragile]{A Tale of Two Loops:}
    \begin{lstlisting}
friends = ['Joseph', 'Glenn', 'Sally']
for friend in friends :
    print('Happy New Year:',  friend)
for i in range(len(friends)) :
    friend = friends[i]
    print('Happy New Year:',  friend)

#output 
Happy New Year: Joseph
Happy New Year: Glenn
Happy New Year: Sally
    \end{lstlisting}
\end{frame}

\begin{frame}{Concatenating Lists Using \texttt{+}}
    \begin{itemize}
        \item We can create a new list by adding two existing lists together
    \end{itemize}
\end{frame}

\begin{frame}[fragile]{Concatenating Lists Example}
    \begin{lstlisting}
>>> a = [1, 2, 3]
>>> b = [4, 5, 6]
>>> c = a + b
>>> print(c)
[1, 2, 3, 4, 5, 6]
>>> print(a)
[1, 2, 3]
    \end{lstlisting}
\end{frame}

\begin{frame}{Lists Can Be Sliced Using \texttt{:}}
    \begin{itemize}
        \item Remember: Just like in strings, the second number is “up to but not including”
    \end{itemize}
\end{frame}

\begin{frame}[fragile]{Slicing Example}
    \begin{lstlisting}
>>> t = [9, 41, 12, 3, 74, 15]
>>> t[1:3]
[41,12]
>>> t[:4]
[9, 41, 12, 3]
>>> t[3:]
[3, 74, 15]
>>> t[:]
[9, 41, 12, 3, 74, 15]
    \end{lstlisting}
\end{frame}

\begin{frame}[fragile]{List Methods}
    \begin{lstlisting}
>>> x = list()
>>> type(x)
<type 'list'>
>>> dir(x)
[... 'append', 'count', 'extend', 'index', 'insert', 'pop', 'remove', 'reverse', 'sort']
>>> 
    \end{lstlisting}
\end{frame}

\begin{frame}{Building a List from Scratch}
    \begin{itemize}
        \item We can create an empty list and then add elements using the append method
        \item The list stays in order and new elements are added at the end of the list
    \end{itemize}
\end{frame}

\begin{frame}[fragile]{Code Example}
    \begin{lstlisting}
>>> stuff = list()
>>> stuff.append('book')
>>> stuff.append(99)
>>> print(stuff)
['book', 99]
>>> stuff.append('cookie')
>>> print(stuff)
['book', 99, 'cookie']
    \end{lstlisting}
\end{frame}

\begin{frame}{Is Something in a List?}
    \begin{itemize}
        \item Python provides two operators that let you check if an item is in a list
        \item These are logical operators that return True or False
        \item They do not modify the list
        \item \texttt{\color{red}in}
        \item \texttt{{\color{red}not in}}
    \end{itemize}
\end{frame}

\begin{frame}[fragile]{Code Example}
    \begin{lstlisting}
>>> some = [1, 9, 21, 10, 16]
>>> 9 in some
True
>>> 15 in some
False
>>> 20 not in some
True
>>> 
    \end{lstlisting}
\end{frame}

\begin{frame}{Lists are in Order}
    \begin{itemize}
        \item A list can hold many items and keeps those items in the order until we do something to change the order
        \item A list can be sorted 
        \irem The \texttt{\color{red}sort} method (unlike in strings) means “sort yourself”
    \end{itemize}
\end{frame}

\begin{frame}[fragile]{Sort Example}
    \begin{lstlisting}
>>> friends = [ 'Joseph', 'Glenn', 'Sally' ]
>>> friends.sort()
>>> print(friends)
['Glenn', 'Joseph', 'Sally']
>>> print(friends[1])
Joseph
>>> 
    \end{lstlisting}
\end{frame}

\begin{frame}{Built-in Functions and Lists}
    \begin{itemize}
        \item There are a number of functions built into Python that take lists as parameters
        \item Remember the loops we built?  These are much simpler.
    \end{itemize}
\end{frame}

\begin{frame}[fragile]{Code Example}
    \begin{lstlisting}
>>> nums = [3, 41, 12, 9, 74, 15]
>>> print(len(nums))
6
>>> print(max(nums))
74
>>> print(min(nums))
3
>>> print(sum(nums))
154
>>> print(sum(nums)/len(nums))
25.6
    \end{lstlisting}
\end{frame}

\begin{frame}{Best Friends: Strings and Lists}
    \begin{itemize}
        \item Split breaks a string into parts and produces a list of strings.  
        \item We think of these as words.  
        \item We can access a particular word or loop through all the words.
    \end{itemize}
\end{frame}

\begin{frame}[fragile]{Code Example}
    \begin{lstlisting}
>>> abc = 'With three words'
>>> stuff = abc.split()
>>> print(stuff)
['With', 'three', 'words']
>>> print(len(stuff))
3
>>> print(stuff[0])
With
>>> print(stuff)
['With', 'three', 'words']
>>> for w in stuff :
...     print(w)
...
With
Three
Words
    \end{lstlisting}
\end{frame}


\begin{frame}{Splitting by Delimiter}
    \begin{itemize}
        \item When you do not specify a delimiter, multiple spaces are treated like one delimiter
       \item You can specify what delimiter character to use in the splitting
    \end{itemize}
\end{frame}

\begin{frame}[fragile]{Code Example}
    \begin{lstlisting}
>>> line = 'A lot               of spaces'
>>> etc = line.split()
>>> print(etc)
['A', 'lot', 'of', 'spaces']
>>>
>>> line = 'first;second;third'
>>> thing = line.split()
>>> print(thing)
['first;second;third']
>>> print(len(thing))
1
>>> thing = line.split(';')
>>> print(thing)
['first', 'second', 'third']
>>> print(len(thing))
3
    \end{lstlisting}
\end{frame}

\begin{frame}[fragile]{Remove Specified Index in List}
    \begin{lstlisting}
thislist = ["apple", "banana", "cherry"]
thislist.pop(1)
print(thislist)

#output
['apple', 'cherry']
    \end{lstlisting}
\end{frame}

\begin{frame}[fragile]{Remove Specified Item in List}
    \begin{lstlisting}
thislist = ["apple", "banana", "cherry"]
thislist.remove("banana")
print(thislist)

#output
['apple', 'cherry']
    \end{lstlisting}
\end{frame}

\begin{frame}
    \begin{center}
        {\Huge\ End of Lists}
    \end{center}
\end{frame}

\end{document}
