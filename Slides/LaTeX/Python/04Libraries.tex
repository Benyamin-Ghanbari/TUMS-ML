\documentclass[serif, aspectratio=169]{beamer}
\usepackage[T1]{fontenc} 
\usepackage{fourier}
\usepackage{hyperref}
\usepackage{latexsym,amsmath,xcolor,multicol,booktabs,calligra}
\usepackage{graphicx,pstricks,listings,stackengine}
\usepackage{listings}

\author{Dr.Hajialiasgari}
\title{Introduction to Python}
\institute{
    Tehran University \\
    Of\\
    Medical Science
}
\date{\small \today}
\usepackage{UoWstyle}

% Define custom colors and styles for listings
\definecolor{deepblue}{rgb}{0,0,0.5}
\definecolor{deepred}{RGB}{153,0,0}
\definecolor{deepgreen}{rgb}{0,0.5,0}
\definecolor{halfgray}{gray}{0.55}

\lstset{
    basicstyle=\ttfamily\small,
    keywordstyle=\bfseries\color{deepblue},
    emphstyle=\ttfamily\color{deepred},
    stringstyle=\color{deepgreen},
    numbers=left,
    numberstyle=\small\color{halfgray},
    rulesepcolor=\color{red!20!green!20!blue!20},
    frame=shadowbox,
}

\begin{document}

\begin{frame}
    \titlepage
    \vspace*{-0.6cm}
    \begin{figure}[htpb]
        \begin{center}
            \includegraphics[keepaspectratio, scale=0.05]{Tumsl-logo.png}
        \end{center}
    \end{figure}
\end{frame}

\begin{frame}    
\tableofcontents[sectionstyle=show, subsectionstyle=show/shaded/hide, subsubsectionstyle=show/shaded/hide]
\end{frame}

\section{Introduction to Python Libraries}

\begin{frame}{What is a Python Library?}
    \begin{itemize}
        \item A Python library is a collection of reusable functions, classes, and modules that provide specific functionality.
        \item Libraries simplify coding by offering pre-built solutions for common tasks.
    \end{itemize}
\end{frame}

\section{Kinds of Python Libraries}

\begin{frame}{1. Standard Libraries}
    \begin{itemize}
        \item Built-in libraries that come with Python and require no installation.
        \item Examples:
        \begin{itemize}
            \item \texttt{math}: Mathematical functions, e.g., \texttt{sqrt}, \texttt{sin}.
            \item \texttt{datetime}: Date and time manipulation.
            \item \texttt{os}: Interacts with the operating system.
            \item \texttt{sys}: System-specific functions and parameters.
            \item \texttt{random}: Generates random numbers.
        \end{itemize}
    \end{itemize}
\end{frame}

\begin{frame}[fragile]{Standard Library Example: \texttt{math}}
    \begin{lstlisting}
import math
print(math.sqrt(16))  # Outputs 4.0
    \end{lstlisting}
\end{frame}

\begin{frame}{2. Third-Party Libraries}
    \begin{itemize}
        \item Libraries developed by others that need to be installed separately.
        \item Installed via package managers like \texttt{pip} or \texttt{conda}.
        \item Examples by domain:
        \begin{itemize}
            \item \textbf{Data Analysis}: \texttt{pandas}, \texttt{numpy}
            \item \textbf{Visualization}: \texttt{matplotlib}, \texttt{seaborn}
            \item \textbf{Machine Learning}: \texttt{scikit-learn}, \texttt{tensorflow}
            \item \textbf{Web Development}: \texttt{flask}, \texttt{django}
        \end{itemize}
    \end{itemize}
\end{frame}

\begin{frame}[fragile]{Installing Libraries}
    \begin{itemize}
        \item Install using \texttt{pip}:
        \begin{lstlisting}
pip install library_name
        \end{lstlisting}
        \item Install using \texttt{conda} (for Anaconda users):
        \begin{lstlisting}
conda install library_name
        \end{lstlisting}
        \item Example:
        \begin{lstlisting}
pip install pandas  # Installs pandas for data analysis
        \end{lstlisting}
    \end{itemize}
\end{frame}

\section{Popular Third-Party Libraries}

\begin{frame}[fragile]{Example: Data Analysis with \texttt{pandas}}
    \begin{lstlisting}
import pandas as pd
df = pd.read_csv('data.csv')
print(df.head())  # Displays the first few rows of the dataset
    \end{lstlisting}
\end{frame}

\begin{frame}[fragile]{Example: Plotting with \texttt{matplotlib}}
    \begin{lstlisting}
import matplotlib.pyplot as plt
plt.plot([1, 2, 3], [4, 5, 6])
plt.show()  # Displays a simple line plot
    \end{lstlisting}
\end{frame}

\section{Conclusion}

\begin{frame}{Conclusion}
    \begin{itemize}
        \item Standard libraries come with Python, while third-party libraries expand functionality.
        \item Python libraries make coding faster and more efficient.
        \item Learning to use popular libraries enhances productivity in data science, machine learning, and web development.
    \end{itemize}
\end{frame}

\end{document}
