%%%%%%%%%%%%%%%%%%%%%%%%%%%%%%%%%%%%%%%%%%%%%%%%%%%%%%
% A Beamer template for University of Wollongong     %
% Based on THU beamer theme                          %
% Author: Qiuyu Lu                                   %
% Date: July 2024                                    %
% LPPL Licensed.                                     %
%%%%%%%%%%%%%%%%%%%%%%%%%%%%%%%%%%%%%%%%%%%%%%%%%%%%%%

\documentclass[serif, aspectratio=169]{beamer}
%\documentclass[serif]{beamer}  % for 4:3 ratio
\usepackage[T1]{fontenc} 
\usepackage{fourier} % see "http://faq.ktug.org/wiki/uploads/MathFonts.pdf" for other options
\usepackage{hyperref}
\usepackage{latexsym,amsmath,xcolor,multicol,booktabs,calligra}
\usepackage{graphicx,pstricks,listings,stackengine}
\usepackage{lipsum}
\usepackage{listings}



\author{Dr.Hajialiasgari}
\title{Introduction To Python}
\institute{
    Tehran University \\
    Of\\
    Medical Science
}
\date{\small \today}
\usepackage{UoWstyle}

% defs
\def\cmd#1{\texttt{\color{red}\footnotesize $\backslash$#1}}
\def\env#1{\texttt{\color{blue}\footnotesize #1}}
\definecolor{deepblue}{rgb}{0,0,0.5}
\definecolor{deepred}{RGB}{153,0,0}
\definecolor{deepgreen}{rgb}{0,0.5,0}
\definecolor{halfgray}{gray}{0.55}

\lstset{
    basicstyle=\ttfamily\small,
    keywordstyle=\bfseries\color{deepblue},
    emphstyle=\ttfamily\color{deepred},    % Custom highlighting style
    stringstyle=\color{deepgreen},
    numbers=left,
    numberstyle=\small\color{halfgray},
    rulesepcolor=\color{red!20!green!20!blue!20},
    frame=shadowbox,
}


\begin{document}

\begin{frame}
    \titlepage
    \vspace*{-0.6cm}
    \begin{figure}[htpb]
        \begin{center}
            \includegraphics[keepaspectratio, scale=0.05]{Tumsl-logo.png}
        \end{center}
    \end{figure}
\end{frame}

\begin{frame}    
\tableofcontents[sectionstyle=show,
subsectionstyle=show/shaded/hide,
subsubsectionstyle=show/shaded/hide]
\end{frame}

\section{Introduction}

\begin{frame}{Why Python ?}
    \begin{itemize}[<+-| alert@+>] % stepwise alerts
        \item \textbf{Introduction to Python}
        \item Python is a powerful, easy-to-learn programming language. Known for its clear syntax, it’s widely used in web development, data analysis, automation, and AI.
        \item Its versatility and extensive libraries make it ideal for beginners and professionals alike.
    \end{itemize}
\end{frame}


\section{Work Environment}
\subsection{Python, VSCode, Jupyter}

\begin{frame}{Installation}
    \begin{itemize}[<+-| alert@+>] % stepwise alerts
        \item \textbf{Install Python:} \newline
        Download and install Python from: \newline
        \url{https://www.python.org/downloads/}
        \item \textbf{Install VSCode:} \newline
        Get Visual Studio Code from: \newline
        \url{https://code.visualstudio.com/}
        \item \textbf{Install Jupyter:} \newline
        Use the command: \newline
        \texttt{pip install jupyterlab} \newline
        More details at: \newline
        \url{https://jupyter.org/install}
    \end{itemize}
\end{frame}



\section{Variables, Expression, Statement}

% Constants Slide
\begin{frame}{Constants}
    \begin{itemize}[<+-| alert@+>] % stepwise alerts
        \item Fixed values such as numbers, letters, and strings, are called “constants” because their value does not change.
        \item Numeric constants are as you expect.
        \item String constants use single quotes (') or double quotes (").
    \end{itemize}
\end{frame}

% Reserved Words Slide
\begin{frame}{Reserved Words}
    \begin{itemize}[<+-| alert@+>] % stepwise alerts
        \item False, await, else, import, pass
        \item None, break, except, in, raise
        \item True, class, finally, is, return
        \item and, continue, for, lambda, try
        \item as, def, from, nonlocal, while
        \item assert, del, global, not, with
        \item async, elif, if, or, yield
    \end{itemize}
\end{frame}

% Variable Slide
\begin{frame}{Variables}
    \begin{itemize}[<+-| alert@+>] % stepwise alerts
        \item A variable is a named place in memory where a programmer can store data and later retrieve it using the variable's name.
        \item Programmers get to choose the names of the variables.
        \item You can change the contents of a variable in a later statement.
    \end{itemize}
\end{frame}

% Python Variable Rules Slide
\begin{frame}{Python Variable Rules}
    \begin{itemize}[<+-| alert@+>] % stepwise alerts
        \item Must start with a letter or an underscore (\_)
        \item Must consist of letters, numbers, and underscores (\_)
        \item Case sensitive (e.g., `var` and `Var` are different variables)
    \end{itemize}
    \vspace{0.5cm}
    \textbf{Examples:}
    \begin{itemize}
        \item Valid: `my\_variable`, `var123`, `\_myVar` \\
        \item Invalid: `123var`, `my-var`, `my var`
    \end{itemize}
\end{frame}

% Different Types Slide
\begin{frame}{Different Types in Python}
    \begin{itemize}[<+-| alert@+>] % stepwise alerts
        \item Python supports different data types, including:
        \begin{itemize}
            \item \textbf{String (str)}: A sequence of characters, e.g., `'Hello'`, `"World"`
            \item \textbf{Integer (int)}: Whole numbers, e.g., `10`, `-5`
            \item \textbf{Float}: Numbers with decimal points, e.g., `3.14`, `-0.001`
            \item \textbf{Complex}: Complex numbers, e.g., `1+2j`
            \item \textbf{Boolean}: True , False
        \end{itemize}
    \end{itemize}
    \vspace{0.5cm}
    \textbf{Examples:}
    \begin{itemize}
        \item String: `"Hello World"`
        \item Integer: `42`
        \item Float: `3.14159`
        \item Complex: `1+2j`
    \end{itemize}
    \textbf{Syntax:}
    \begin{itemize}
        \item type (42)\\
<class ’int’>
    \end{itemize}
    
\end{frame}

\section{Numeric Expression}

% Operators and Operations Slide
\begin{frame}{Operators and Operations}
    \begin{tabular}{|l|l|}
        \hline
        \textbf{Operator} & \textbf{Operation} \\
        \hline
        \texttt{+} & Addition \\
        \texttt{-} & Subtraction \\
        \texttt{*} & Multiplication \\
        \texttt{/} & Division \\
        \texttt{**} & Exponentiation \\
        \texttt{//} & Floor Division \\
        \texttt{\%} & Modulus (Remainder) \\
        \hline
    \end{tabular}
\end{frame}

% Example of Operations Slide 1
\begin{frame}{Examples of Operations}
    \textbf{Addition and Subtraction:}
    \begin{itemize}[<+-| alert@+>]
        \item \texttt{5 + 3 = 8}
        \item \texttt{10 - 4 = 6}
    \end{itemize}
\end{frame}

% Example of Operations Slide 2
\begin{frame}{Examples of Operations}
    \textbf{Multiplication and Division:}
    \begin{itemize}[<+-| alert@+>]
        \item \texttt{4 * 3 = 12}
        \item \texttt{8 / 2 = 4.0}
    \end{itemize}
\end{frame}

% Example of Operations Slide 3
\begin{frame}{Examples of Operations}
    \textbf{Exponentiation, Floor Division, Modulus:}
    \begin{itemize}[<+-| alert@+>]
        \item \texttt{2 ** 3 = 8}
        \item \texttt{7 // 2 = 3}
        \item \texttt{7 \% 2 = 1}
    \end{itemize}
\end{frame}

% Order of Evaluation Slide
\begin{frame}{Order of Evaluation}
    \begin{itemize}[<+-| alert@+>]
        \item When we string operators together, Python must know which one to do first.
        \item This is called “operator precedence.”
        \item Which operator “takes precedence” over the others?
    \end{itemize}
\end{frame}

% Operator Precedence Rule Slide
\begin{frame}{Operator Precedence Rule}
    \textbf{Highest precedence rule to lowest precedence rule:}
    \begin{itemize}[<+-| alert@+>]
        \item Parentheses are always respected: \texttt{(2 + 3) * 4 = 20}
        \item Exponentiation (raise to a power): \texttt{2 ** 3 = 8}
        \item Multiplication, Division, and Remainder: \texttt{8 / 2 * 4 = 16}
        \item Addition and Subtraction: \texttt{3 + 5 - 2 = 6}
        \item Left to right: \texttt{2 + 3 * 4 = 14}
    \end{itemize}
\end{frame}

% User Input Slide
\begin{frame}[fragile]{User Input}
    \begin{itemize}[<+-| alert@+>]
        \item We can instruct Python to pause and read data from the user using the \texttt{input()} function.
        \item The \texttt{input()} function returns a string.
    \end{itemize}
    \vspace{0.5cm}
    \textbf{Example:}
    \begin{verbatim}
name = input("Enter your name: ")
print("Hello, " + name)
    \end{verbatim}
\end{frame}
% Converting Input Slide
\section{Conditional Execution, Input}

\subsection{Converting Input, if else}

% Converting Input Slide
\begin{frame}[fragile]{Converting Input}
    \begin{itemize}[<+-| alert@+>]
        \item If we want to read a number from the user, we must convert it from a string to a number using a type conversion function.
        \item Later we will deal with bad input data.
    \end{itemize}
    \vspace{0.5cm}
    \textbf{Example:}
    \begin{verbatim}
age = input("Enter your age: ")
age = int(age)
print(age + 5)
    \end{verbatim}
\end{frame}

% Conditional Execution Slide
% Conditional Execution Slide
\begin{frame}{Conditional Execution}
    \begin{itemize}[<+-| alert@+>] % stepwise alerts
        \item Conditional execution allows the program to execute certain blocks of code based on specific conditions.
        \item The primary conditional statements in Python are \texttt{if}, \texttt{elif}, and \texttt{else}.
    \end{itemize}
\end{frame}

% Comparison Operators Slide
\begin{frame}{Comparison Operators}
    \begin{tabular}{|l|l|l|}
        \hline
        \textbf{Operator} & \textbf{Description} & \textbf{Example} \\
        \hline
        \texttt{==} & Equal to & \texttt{a == b} \\
        \texttt{!=} & Not equal to & \texttt{a != b} \\
        \texttt{>} & Greater than & \texttt{a > b} \\
        \texttt{<} & Less than & \texttt{a < b} \\
        \texttt{>=} & Greater than or equal to & \texttt{a >= b} \\
        \texttt{<=} & Less than or equal to & \texttt{a <= b} \\
        \hline
    \end{tabular}
\end{frame}


% If Statement Slide
\begin{frame}[fragile]{If Statement}
    \begin{itemize}[<+-| alert@+>] % stepwise alerts
        \item The \texttt{if} statement checks a condition and executes the block of code if the condition is true.
    \end{itemize}
    \vspace{0.5cm}
    \textbf{Example:}
    \begin{verbatim}
if age >= 18:
    print("You are an adult")
    \end{verbatim}
\end{frame}

% Else Statement Slide
\begin{frame}[fragile]{Else Statement}
    \begin{itemize}[<+-| alert@+>] % stepwise alerts
        \item The \texttt{else} statement executes if all preceding conditions are false.
    \end{itemize}
    \vspace{0.5cm}
    \textbf{Example:}
    \begin{verbatim}
if temperature > 30:
    print("It's a hot day.")
else:
    print("It's a pleasant day.")
    \end{verbatim}
\end{frame}

% Nested Conditional Statements Slide
\begin{frame}[fragile]{Nested Conditionals}
    \begin{itemize}[<+-| alert@+>] % stepwise alerts
        \item You can nest \texttt{if} statements within other \texttt{if} statements to create more complex conditions.
    \end{itemize}
    \vspace{0.5cm}
    \textbf{Example:}
    \begin{verbatim}
if score >= 50:
    print("You passed.")
    if score >= 90:
        print("Excellent!")
    else:
        print("Good job.")
else:
    print("You failed.")
    \end{verbatim}
\end{frame}


% End of Session Slide
\begin{frame}
    \begin{center}
        {\Huge\ End of session 1}
    \end{center}
\end{frame}
\end{document}