\documentclass[serif, aspectratio=169]{beamer}
%\documentclass[serif]{beamer}  % for 4:3 ratio
\usepackage[T1]{fontenc} 
\usepackage{fourier}
\usepackage{hyperref}
\usepackage{latexsym,amsmath,xcolor,multicol,booktabs,calligra}
\usepackage{graphicx,pstricks,listings,stackengine}
\usepackage{listings}

\author{Dr.Hajialiasgari}
\title{Introduction To Python}
\institute{
    Tehran University \\
    Of\\
    Medical Science
}
\date{\small \today}
\usepackage{UoWstyle}

% defs
\def\cmd#1{\texttt{\color{red}\footnotesize $\backslash$#1}}
\def\env#1{\texttt{\color{blue}\footnotesize #1}}
\definecolor{deepblue}{rgb}{0,0,0.5}
\definecolor{deepred}{RGB}{153,0,0}
\definecolor{deepgreen}{rgb}{0,0.5,0}
\definecolor{halfgray}{gray}{0.55}

\lstset{
    basicstyle=\ttfamily\small,
    keywordstyle=\bfseries\color{deepblue},
    emphstyle=\ttfamily\color{deepred},
    stringstyle=\color{deepgreen},
    numbers=left,
    numberstyle=\small\color{halfgray},
    rulesepcolor=\color{red!20!green!20!blue!20},
    frame=shadowbox,
}

\begin{document}

\begin{frame}
    \titlepage
    \vspace*{-0.6cm}
    \begin{figure}[htpb]
        \begin{center}
            \includegraphics[keepaspectratio, scale=0.05]{Tumsl-logo.png}
        \end{center}
    \end{figure}
\end{frame}

\begin{frame}    
\tableofcontents[sectionstyle=show, subsectionstyle=show/shaded/hide, subsubsectionstyle=show/shaded/hide]
\end{frame}

\section{Loops and Iteration}

\begin{frame}{Repeated Steps}
    Loops (repeated steps) have iteration variables that change each time through a loop. Often these iteration variables go through a sequence of numbers.
\end{frame}

\subsection{While Loops}

\begin{frame}[fragile]{An Infinite Loop}
    \begin{verbatim}
n = 5
while n > 0 :
    print('Lather')
    print('Rinse')
print('Dry off!')
    \end{verbatim}
    \textcolor{red}{What is wrong with this loop?}
\end{frame}

\begin{frame}[fragile]{Breaking Out of the Loop (Code)}
    The \texttt{break} statement ends the current loop and jumps to the statement immediately following the loop.

    \begin{lstlisting}
while True:
    line = input('> ')
    if line == 'done' :
        break
    print(line)
print('Done!')
    \end{lstlisting}
\end{frame}

\begin{frame}[fragile]{Breaking Out of the Loop (Output)}
    \textbf{Output:}
    \begin{verbatim}
> hello there
hello there
> finished
finished
> done
Done!
    \end{verbatim}
\end{frame}

\begin{frame}[fragile]{Finishing an Iteration with continue (Code)}
    The \texttt{continue} statement ends the current iteration and jumps to the top of the loop to start the next iteration.

    \begin{lstlisting}
while True:
    line = input('> ')
    if line[0] == '#' :
        continue
    if line == 'done' :
        break
    print(line)
print('Done!')
    \end{lstlisting}
\end{frame}

\begin{frame}[fragile]{Finishing an Iteration with continue (Output)}
    \textbf{Output:}
    \begin{verbatim}
> hello there
hello there
> # don't print this
> print this!
print this!
> done
Done!
    \end{verbatim}
\end{frame}

\subsection{For Loops}

\begin{frame}[fragile]{Definite Loops}
    \begin{center}
        Iterating over a set of items…
    \end{center}

    \begin{itemize}
        \item Loops that iterate over a finite set of things are called “definite loops.”
        \item Example:
    \end{itemize}

    \begin{lstlisting}
for i in range(1,10) :
    print(i)
print('Your first loop!')
    \end{lstlisting}
\end{frame}

\begin{frame}[fragile]{Finding the Average in a Loop (Code)}
    \begin{lstlisting}
count = 0
sum = 0
print('Before', count, sum)
for value in [9, 41, 12, 3, 74, 15] :
    count = count + 1
    sum = sum + value
    print(count, sum, value)
    \end{lstlisting}
\end{frame}

\begin{frame}[fragile]{Filtering in a Loop (Code)}
    \begin{lstlisting}
print('Before')
for value in [9, 41, 12, 3, 74, 15] :
    if value > 20:
        print('Large number', value)
print('After')
    \end{lstlisting}

    \textcolor{red}{We use an if statement in the loop to catch/filter the values we are looking for.}
\end{frame}

\begin{frame}[fragile]{Finding the Smallest Value (Code)}
    \begin{lstlisting}
smallest = None
for value in [9, 41, 12, 3, 74, 15] :
    if smallest is None : 
        smallest = value
    elif value < smallest : 
        smallest = value
    print(smallest, value)
print(smallest)
    \end{lstlisting}

    We still have a variable that is the smallest so far. The first time through the loop, smallest is \texttt{None}, so we take the first value to be the smallest.
\end{frame}

\begin{frame}[fragile]{The is and is not Operators}
   \begin{lstlisting}
smallest = None
for value in [3, 41, 12, 9, 74, 15] :
    if smallest is None : 
        smallest = value
    elif value < smallest : 
        smallest = value
    print(smallest, value)
   \end{lstlisting}
   
   Python has an \texttt{is} operator that can be used in logical expressions. \\
   Implies “is the same as” \\
   Similar to, but stronger than, \texttt{==}. \\
   \texttt{is not} also is a logical operator.
\end{frame}


% End of Session Slide
\begin{frame}
    \begin{center}
        {\Huge\ End of Loops and Iteration}
    \end{center}
\end{frame}

\end{document}
